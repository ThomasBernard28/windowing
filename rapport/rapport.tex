\documentclass{article}
\usepackage{svg}
\usepackage{amsmath}
\usepackage[utf8]{inputenc}
\usepackage[french]{babel}
\usepackage{graphicx}
\usepackage[T1]{fontenc}
\usepackage{hyperref}
\hypersetup{
    colorlinks,
    citecolor=black,
    filecolor=black,
    linkcolor=black,
    urlcolor=black
}
\graphicspath{images/}

\usepackage{blindtext}

\usepackage{subfiles} % Best loaded last in the preamble

\title{Rapport: Exercice préliminaire}
\author{ }
\date{ }

\begin{document}
\begin{titlepage}
    \begin{center}
        
        {\Large Université de Mons}\\[1ex]
        {\Large Faculté des sciences}\\[1ex]
        {\Large Département d'Informatique}\\[2.5cm]
        
        \newcommand{\HRule}{\rule{\linewidth}{0.3mm}}
        % Title
        \HRule \\[0.3cm]
        { \LARGE \bfseries Stuctures de données II \\[0.3cm]}
        { \LARGE \bfseries Rapport de l'exercice préléminaire \\[0.1cm]} % Commenter si pas besoin
        \HRule \\[1.5cm]
        
        % Author and supervisor
        \begin{minipage}[t]{0.45\textwidth}
            \begin{flushleft} \large
                \emph{Professeur:}\\
                Véronique \textsc{Bruyère}\\
                \emph{Assistant:}\\
                Pierre \textsc{Hauweele}\\
            \end{flushleft}
        \end{minipage}
        \begin{minipage}[t]{0.45\textwidth}
            \begin{flushright} \large
                \emph{Auteurs:} \\
                Theo \textsc{Godin} \\
                Thomas \textsc{Bernard} 
            \end{flushright}
        \end{minipage}\\[2ex]
        
        \vfill
        
        % Bottom of the page
        \begin{center}
            \begin{tabular}[t]{c c c}
                \includegraphics[height=1.5cm]{images/logoumons.jpg} &
                \hspace{0.3cm} &
                \includegraphics[height=1.5cm]{images/logofs.jpg}:
            \end{tabular}
        \end{center}~\\
        
        {\large Année académique 2022-2023}
        
    \end{center}
\end{titlepage}

\tableofcontents

\newpage

\section{Question 1}
    On voit en effet qu'il s'agit d'une file à priorité sur les coordonnées x. On va toujours prendre le point possédant la coordonnée x minimum
    afin de le placer dans la racine et ce pour chaque racine de chaque sous arbre. Ce qui fait que chaque racine de chaque sous arbre est un x minimal du
    jeux de points restants pour le sous arbre étudié. Donc, chacun des fils a bien une coordonée x inférieure ou égale à celle de son père.
    En revanche, il ne s'agit pas d'un tas. En effet, les couches ne sont pas remplies de droite à gauche et la condition de répartition établie
    sur la médiane des y impose qu'il y ait autant de noeuds dans les sous arbres gauche et droit de la racine. En conclusion, aucune condition
    n'impose de remplir la dernière couche complètement de droite à gauche si il n'y a pas assez de noueds pour le faire. On ne peut donc pas avoir de tas.\\
    La coordonnée x minimum se trouve donc dans la racine de l'arbre et la 
    coordonnée x maximum se trouve dans une des feuilles.

\section{Question 2}
    On voit qu'il s'agit d'une arbre binaire de recherche. Tout d'abord, les conditions d'un ABR sont que pour tout noeud n, son fils gauche $f_g$ est 
    tel que $n.data > f_g.data$ et son fils droit $f_d$ est tel que $n.data < f_d.data$. \\
    Ici la "data" qui est utilisée pour comparer chaque noeud avec ses fils avant une insertion est $y_{mid}$ la médiane des coordonnées y des points
    présents dans l'esemble actuellement considéré. On retrouvera donc dans le sous arbre gauche de la racine l'ensemble des points dont les coordonnées
    y sont strictements inférieures à la médiane $y_{mid}$. Par conséquent la médiane des coordonnées y des points du sous arbre gauche qui servira
    de "data" pour le fils gauche de la racine sera également strictement inférieure à la médiane $y_{mid}$. Pour le sous arbre droit il s'agit du même raisonnement
    mais en partant du fait que les coordonnées y des points du sous arbre droit seront strictement supérieures à la médiane $y_{mid}$, id. pour la médiane des coordonnée
    y des ces points. \\
    La coordonnée y minimum se trouve donc dans la feuille la plus à gauche de l'ABR comme elle aura été strictement inférieure à toutes les médianes elle-même inférieures
    à celle de la racine. Et la coordonnée maximum se trouvera dans la feuille la plus à droite de l'ABR comme elle aura été strictement suppérieure à toutes
    les médianes elles-même supérieures à celle de la racine.

\newpage
\section{Question 3}
    Le fait que les noeuds soient répartis dans les sous-arbres droit et gauche de la racine à l'aide d'une médiane assure par la définition de
    la médiane qu'il y aura autant de noeuds possédant une coordonnées $y < y_{mid}$ que de noeuds possédant une coordonnée $y > y_{mid}$.
    Ce principe est respecté de manière récursive pour chaque noeud de chaque sous arbre. On peut donc voir l'équilibre de cette structure comme le fait
    qu'il y ait autant de noeuds à droite qu'à gauche de chaque racine de chaque sous arbre.

\section{Question 4}

\section{Question 5}

\section{Question 6}

\section{Question 7}

\section{Question 8}

\end{document}
