\documentclass{article}
\usepackage{svg}
\usepackage{amsmath}
\usepackage[utf8]{inputenc}
\usepackage[french]{babel}
\usepackage{graphicx}
\usepackage[T1]{fontenc}
\usepackage{hyperref}
\hypersetup{
    colorlinks,
    citecolor=black,
    filecolor=black,
    linkcolor=black,
    urlcolor=black
}
\graphicspath{images/}

\usepackage{blindtext}

\usepackage{subfiles} % Best loaded last in the preamble

\title{Rapport: Exercice préliminaire}
\author{ }
\date{ }

\begin{document}
\begin{titlepage}
    \begin{center}
        
        {\Large Université de Mons}\\[1ex]
        {\Large Faculté des sciences}\\[1ex]
        {\Large Département d'Informatique}\\[2.5cm]
        
        \newcommand{\HRule}{\rule{\linewidth}{0.3mm}}
        % Title
        \HRule \\[0.3cm]
        { \LARGE \bfseries Stuctures de données II \\[0.3cm]}
        { \LARGE \bfseries Rapport de l'exercice préléminaire \\[0.1cm]} % Commenter si pas besoin
        \HRule \\[1.5cm]
        
        % Author and supervisor
        \begin{minipage}[t]{0.45\textwidth}
            \begin{flushleft} \large
                \emph{Professeur:}\\
                Véronique \textsc{Bruyère}\\
                \emph{Assistant:}\\
                Pierre \textsc{Hauweele}\\
            \end{flushleft}
        \end{minipage}
        \begin{minipage}[t]{0.45\textwidth}
            \begin{flushright} \large
                \emph{Auteurs:} \\
                Theo \textsc{Godin} \\
                Thomas \textsc{Bernard} 
            \end{flushright}
        \end{minipage}\\[2ex]
        
        \vfill
        
        % Bottom of the page
        \begin{center}
            \begin{tabular}[t]{c c c}
                \includegraphics[height=1.5cm]{images/logoumons.jpg} &
                \hspace{0.3cm} &
                \includegraphics[height=1.5cm]{images/logofs.jpg}:
            \end{tabular}
        \end{center}~\\
        
        {\large Année académique 2022-2023}
        
    \end{center}
\end{titlepage}

\tableofcontents

\newpage

\section{Question 1}
    Oui on voit qu'il s'agit d'une file de priorité. La coordonnée x minimum se trouve à la racine alors que la coordonnéex maximum se trouve quant à elle
    dans une des feuilles. On peut ainsi accéder à une fenêtre de coordonnée sur x en partant de son minimum en allant jusqu'à son maximum une feuille. 
    On peut voir qu'il s'agit d'un tas car tout noeud père a une coordonnée x inférieure à celles de ses fils (droits et gauches). La contrainte disant
    qu'il doit y avoir autant de noeuds dans la partie droite que la partie gauche correspond bien à la définition d'un tas qui veut que chaque couche
    soit remplie à l'exception de la couche où se trouvent les feuilles.

\section{Question 2}
    Oui on voit bien que les données sont organisées sous forme d'un arbre binaire car une coordonée y inférieur à celle de son père
    deviendra le fils gauche de ce père alors que qu'elle deviendra le fils droit de ce même père dans l'autre cas. La coordonnée minimale se trouve donc dans la feuille la plus à gauche 
    alors que la coordonnée maximale se trouve elle dans la feuille la plus à droite.

\section{Question 3}
    L'arbre de recherche à priorité est équilibré grâce au fait que l'on se serve de la médiane des coordonnées y pour répartir les noeuds dans les sous arbres
    gauches et droits de la racine. Cela assure une certaine parité au niveau du nombre de noeuds présents dans les sous arbres gauches et droits de la racine.
    De plus le modèle de tas pour les coordonnées x assure également cette parité par rapport aux coordonnées x qui ont seulement pour contraintes d'être
    supérieures à celles de leur père.

\section{Question 4}

\section{Question 5}

\section{Question 6}

\section{Question 7}

\section{Question 8}

\end{document}
